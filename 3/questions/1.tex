\smalltitle{سوال 1}\\
\smalltitle{قسمت اول}
\begin{enumerate}
    \item \hfill \codesample{codes/1-1.sql}
    \item \hfill \codesample{codes/1-2.sql}
    \item \hfill \codesample{codes/1-3.sql}
    \item \hfill \codesample{codes/1-4.sql}
    \item در ابتدا لیست پرستار‌هایی را پیدا می‌کنیم که ثبت نام شده باشند. این کار را صرفا با یک کوئری ساده
    \lr{select}
    می‌توان انجام داد. سپس این مقادیر را در جدول
    \lr{Appointment}، \lr{semi join}پ
    می‌زنیم که فقط ملاقات‌هایی بمانند که پرستار‌های آنها ثبت نام شده باشند.
    سپس به کمک
    \lr{group by} و \lr{having}
    آن بیمارانی را پیدا می‌کنیم که حداقل دو قرار ملاقات با این مشخصات دارند. کوئری که تنها اسم بیماران را دهد
    به صورت زیر است:
    \codesample{codes/1-5.sql}
    \noindent
    اما به نظر میرسید که سوال نام پزشک‌ها را نیز می‌خواهد. برای بدست آوردن این موضوع، بعد از در آوردن
    \lr{SSN}
    بیماران، دوباره لیست قرار ملاقات‌های مورد نظر را در می‌آوریم و این بار نام پزشک را
    در کنار نام بیمار قرار می‌دهیم.
    \codesample{codes/1-5-2.sql}
    \item در ابتدا پزشک را با
    \lr{Trained\_in}
    جوین می‌کنیم که کنار هر پزشک شناسه‌ی عملی که برای آن آموزش دیده است را بدست آوریم.
    سپس جدول حاصل را با جدول
    \lr{Undergoes}
    جوین می‌کنیم. سپس در یک جدول شناسه پزشک، شناسه عمل پزشکی و شناسه عملی که دکتر در آن تخصص دارد را داریم.
    سپس آن عمل‌هایی را پیدا می‌کنیم که شناسه عمل پزشکی با شناسه عملی که دکتر در آن تخصص دارد متفاوت باشد.
    در نهایت نیز با جوین دادن جدول با جدول پزشک‌ها و جدول عمل‌های پزشکی می‌توانیم نام پزشک و نام عمل را بدست
    آوریم.
    \codesample{codes/1-6.sql}
    \item \hfill \codesample{codes/1-7.sql}
    \item \hfill \codesample{codes/1-8.sql}
\end{enumerate}
\smalltitle{قسمت دوم}
\begin{enumerate}
    \item \hfill \codesample{codes/1-9.sql}
    \item \hfill \codesample{codes/1-10.sql}
    % https://stackoverflow.com/a/15850510/4213397
    \item \hfill \codesample{codes/1-11.sql}
\end{enumerate}

