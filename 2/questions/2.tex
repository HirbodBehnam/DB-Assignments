\smalltitle{سوال 2}
\begin{enumerate}
    \item $\text{مشتری} \div \Pi_{\text{شهر ، خیابان}}(\sigma_{\text{نام مشتری} = \text{رضا}}(\text{مشتری}))$
    \item $\Pi_{\text{وام گیرنده}}(\text{وام} \div \rho_{\text{شعب}(\text{شعبه وام دهنده})} (\Pi_{\text{نام شعبه}}(\text{شعبه})))$
    \item در ابتدا قسمت دوم سوال را در می‌آوریم. یعنی نام و نام شهر مشتریانی که در همه‌ی شعب تهران حساب دارند.
    برای این کار در ابتدا لیست تمامی شعب تهران را پیدا می‌کنیم و بر لیست افراد و شعبه بانکی تقسیم می‌کنیم.
    برای این کار از رابطه‌های زیر استفاده می‌کنیم:
    \begin{gather*}
        \text{CustomerAccountPlaces} := \Pi_{\text{نام مشتری، شهر مشتری، شهر شعبه}}(
        \\
        (\rho_{\text{شعبه}(\text{نام شعبه، شهر شعبه})}\text{شعبه})
        \\
        \bowtie_{\text{شعبه افتتاح کننده}=\text{نام شعبه}}
        \\
        (\text{حساب}
        \\
        \bowtie_{\text{نام مشتری}=\text{نام صاحب حساب}}
        \\(\rho_{\text{مشتری}(\text{نام مشتری، شهر مشتری})}(\Pi_{\text{نام مشتری، شهر}}(\text{مشتری})))))
        \\
        A := \text{CustomerAccountPlaces} \div \Pi_{\text{شهر شعبه}}(\rho_{\text{شعب}(\text{نام شعبه، شهر شعبه})}(\sigma_{\text{شهر}=\text{تهران}}\text{شعبه}))
    \end{gather*}
    بعد از انجام رابطه بالا،
    $A$
    شامل دو ستون نام مشتری و شهر مشتری است که حاوی مشتریانی است که در همه‌ی شعب در تهران حساب دارند.
    حال برای حساب کردن قسمت اول سوال بدین صورت عمل می‌کنیم؛ در ابتدا باید افرادی را پیدا کنیم که به اندازه‌ی
    تعداد شعبی که در شهر خودشان نیست وام گرفته باشند. با این کار یک سری کاندید پیدا می‌کنیم که ممکن است
    جواب مسئله ما باشند. حال از این افراد آنهایی را حذف می‌کنیم که حداقل یک وام از شهر خود دارند.
    در ابتدا تعداد شعب مختلفی که افراد مختلف از آن وام گرفته‌اند را حساب می‌کنیم.
    برای سادگی من از
    \lr{natural join}
    استفاده می‌کنم که رابطه کوتاه‌تر شود.
    \begin{gather*}
        \text{LoanBranchCount} := \rho_{\text{وام‌ شمار}(\text{نام مشتری، تعداد شعب وام})} (\text{نام وام گیرنده} \mathcal{G}_{\operatorname{count}(\text{شعبه وام دهنده})}(\text{وام}))
    \end{gather*}
    حال تعداد شعب هر شهر را نیز بدست می‌آوریم:
    \begin{gather*}
        \text{BranchesInCity} := \rho_{\text{شعبه شمار}(\text{شهر، تعداد شعب})} (\text{شهر} \mathcal{G}_{\operatorname{count}(\text{نام شعبه})}(\text{شعبه}))
    \end{gather*}
    سپس با جوین کردن وام شمار با جدول مشتری شهر هر مشتری را بدست می‌آوریم. در نهایت با جوین کردن جدول حاصل با
    \lr{BranchesInCity}
    می‌توانیم پیدا کنیم که در شهر هر کاربر چند شعبه وجود دارد. در نهایت نیز جدول حاصل را در تعداد کل شعب
    ضرب دکارتی می‌کنیم. (کوئری در زیر آمده است)
    \begin{gather*}
        \text{TotalBranches} := \mathcal{G}_{\operatorname{count}(\text{نام شعبه})}(\text{شعبه})\\
        \text{CandidateCustomers} :=
        \Pi_{\text{نام مشتری}} \bigl(
        \sigma_{\text{تعداد شعب وام} + \text{تعداد شعب شهر مشتری} = \text{Total Branches}} \bigl(
        \\
        \Pi_{\text{نام مشتری، تعداد شعب وام، تعداد شعب شهر مشتری}} (\text{LoanBranchCount} \infty \text{مشتری} \infty \text{BranchesInCity})
        \\
        \times
        \\
        \text{TotalBranches}
        \bigr) \bigr)
    \end{gather*}
    حال باید مشتری‌هایی را از این لیست پیدا کنیم که از شهر خود وام گرفته باشند؛ پاسخ این قسمت برابر متمم پاسخ
    کل این قسمت است.
    \begin{gather*}
        \text{NotAnswer} := \Pi_{\text{نام مشتری، نام شهر مشتری}} \bigl( \sigma_{\text{شهر شعبه} = \text{شهر مشتری}} ((\text{مشتری} \ltimes \text{CandidateCustomers}) \infty \text{وام} \infty \text{شعبه})\bigr)\\
        \text{B} := \Pi_{\text{نام مشتری، نام شهر مشتری}}(\text{مشتری}) - \text{NotAnswer}
    \end{gather*}
    در نهایت
    $A$
    و
    $B$
    را اشتراک می‌گیریم:
    \begin{gather*}
        A \cap B
    \end{gather*}
    \item در ابتدا لیست همشهری‌های مجید را پیدا می‌کنیم. برای اینکار در ابتدا شهر مجید را پیدا می‌کنیم. خروجی
    مورد نظر یک جدول است که یک ستون و یک سطر دارد. سپس این جدول را با خود مشتری‌ها
    \lr{join}
    می‌دهیم. با این کار اسامی افرادی در می‌آید که همشهری مجید هستند و خود مجید در می‌آید. معلوم نیست که در سوال
    لازم است که خود مجید را همشهری مجید بگیریم یا خیر. در صورتی که خود مجید مد نظر نیست، کافی است که از
    $-\{\text{مجید}\}$
    استفاده کنیم در آخر رابطه زیر:
    \begin{gather*}
        \text{MajidFellowCitizen} := \Pi_{\text{نام مشتری}} (\text{مشتری} \infty (\Pi{\text{شهر}}(\sigma_{\text{نام مشتری}=\text{مجید}}\text{مشتری})))
    \end{gather*}
    حال می‌خواهیم لیست شعبی را انتخاب کنیم که به این مشتری‌ها وام داده‌اند. برای این کار، لیست مشتریان را
    با لیست وام‌ها
    \lr{join}
    می‌دهیم.
    \begin{gather*}
        \text{BanksLoanToMajidFellowCitizen} := \Pi_{\text{نام شعبه}}(\text{MajidFellowCitizen} \bowtie_{\text{نام مشتری}=\text{وام گیرنده}} \text{وام})
    \end{gather*}
    حال باید لیست شعبه‌هایی را پیدا کنیم که هیچ کدام از همشهری‌های رضا در آن حساب ندارند. در ابتدا مثل قسمت
    اول لیست همشهری‌های رضا را پیدا می‌کنیم. سپس لیست شعبه‌هایی را پیدا می‌کنیم که همشهری‌های رضا در آن حساب
    \underline{دارند}.
    در نهایت تمام شعب را منهای جدول بدست آمده می‌کنیم.
    \begin{gather*}
        \text{RezaFellowCitizen} := \Pi_{\text{نام مشتری}} (\text{مشتری} \infty (\Pi{\text{شهر}}(\rho_{\text{نام مشتری}=\text{رضا}}\text{مشتری})))
        \\
        \text{BanksWithAccountsOfRezaFellowCitizens} :=\\ \Pi_{\text{نام شعبه}}(\text{RezaFellowCitizen} \bowtie_{\text{نام مشتری}=\text{نام صاحب حساب}} \text{حساب})
        \\
        \text{BanksWithoutAccountsOfRezaFellowCitizens} :=\\ \Pi_{\text{نام شعبه}} \text{شعبه} - \text{BanksWithAccountsOfRezaFellowCitizens}
    \end{gather*}
    در نهایت دو مقدار حاصل را از هم کم می‌کنیم.
    \begin{gather*}
        \text{BanksLoanToMajidFellowCitizen} - \text{BanksWithoutAccountsOfRezaFellowCitizens}
    \end{gather*}
\end{enumerate}
