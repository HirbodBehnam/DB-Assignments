\smalltitle{سوال 1}\\
همان طور که در کلاس نیز گفته شده بود، در ابتدا موجودیت‌های قوی، سپس موجودیت‌های ضعیف، بعد رابطه‌های چند به چند
و در نهایت ارث بری‌ها را مدل می‌کنیم. من یک تغییر خیلی ریزی در
\lr{ER}
دادم که هر
\lr{Owner}
باید یک
\lr{ID}
داشته باشد. با این فرض‌ها رابطه‌های ما به صورت زیر هستند:
\begin{latin}
\begin{itemize}
    % Strong entities
    \item PLANE\_TYPE (\underline{Model}, Capacity, Weight)
    \item AIRPLANE (\underline{Reg\#}, Type, Stored\_In)
    \item HANGER (\underline{Number}, Capacity, Location)
    % Weak entities
    \item SERVICE (\underline{Date}, \underline{Workcode}, \underline{Plane}, Hours)
    % Superclasses
    \item PERSON (\underline{Ssn}, Name, Address, Phone)
    \item CORPORATION (\underline{Name}, Address, Phone)
    % Subclasses
    \item OWNER (\underline{ID}, Ssn, Name, Type)
    \item PILOT (\underline{Ssn}, Restr, Lic\_num)
    \item EMPLOYEE (\underline{Ssn}, Salary, Shift)
    % M:N relations
    \item AIRPLANE\_OWNERS (\underline{Reg\#}, \underline{Owner\_ID}, Pdate)
    \item FLIES (\underline{Model}, \underline{Ssn})
    \item WORKS\_ON (\underline{Model}, \underline{Ssn})
\end{itemize}
\end{latin}
\noindent
حال کلید‌های خارجی را مشخص می‌کنیم:
\begin{latin}
\begin{itemize}
    \item AIRPLANE(Type) References PLANE\_TYPE(Model)
    \item AIRPLANE(Stored\_In) References HANGER(Number)
    \item SERVICE(Plane) References AIRPLANE(Reg\#)
    \item OWNER(Ssn) References PERSON(Ssn)
    \item OWNER(Name) References CORPORATION(Name)
    \item PILOT(Ssn) References PERSON(Ssn)
    \item EMPLOYEE(Ssn) References PERSON(Ssn)
    \item AIRPLANE\_OWNERS(Reg\#) References AIRPLANE(Reg\#)
    \item AIRPLANE\_OWNERS(Owner\_ID) References OWNER(ID)
    \item FLIES(Model) References PLANE\_TYPE(Model)
    \item FLIES(Ssn) References PILOT(Model)
    \item WORKS\_ON(Model) References PLANE\_TYPE(Model)
    \item WORKS\_ON(Ssn) References EMPLOYEE(Model)
\end{itemize}
\end{latin}
\noindent
قبل از اینکه ویژگی‌های
\lr{nullable}
را نام ببرم می‌خواستم بگویم که به نظرم کمی مدل
\lr{ER}
مشکل رابطه‌های اجباری دارد. به عنوان مثال به نظر من باید حتما
\lr{Airplane}
به
\lr{Plane Type}
رابطه اجباری داشته باشد ولی در مدل
\lr{ER}
این طوری نیست. به هر حال با اعمال این روابط، ویژگی‌های
\lr{nullable}
به صورت زیر هستند:
\begin{latin}
\begin{itemize}
    \item SERVICE(Hours)
    \item OWNER(Ssn)
    \item OWNER(Name)
\end{itemize}
\end{latin}
برای یونیک بودن نیز به نظرم این صفت‌ها + کلید‌های اصلی باید یونیک باشند:
\begin{latin}
\begin{itemize}
    \item PILOT(Lic\_num)
\end{itemize}
\end{latin}

