\smalltitle{سوال 2}
\begin{enumerate}
    \item در ابتدا داخلی ترین \lr{sql}
    را بررسی می‌کنیم که جلوی
    \lr{from}
    است. این کوئری تمامی جداول را جوین می‌دهد و می‌گوید که هر کاربر چند کتاب را به امانت برده‌است که ناشر آنها
    \lr{springer}
    باشد. در جلوی
    \lr{where}
    نیز ما تعداد کتاب‌هایی را پیدا می‌کنیم که ناشر آنها
    \lr{springer}
    باشد. این کوئری تا حدود خیلی خوبی درست است. کافی است که
    \codeword{count (Book.ISBN)}
    را به
    \codeword{count (distinct Book.ISBN)}
    تغییر دهیم. چرا که در غیر این حالت اگر فردی 10 بار یک کتاب را امانت گرفته بود، 10 بار آنرا حساب
    می‌کردیم.
    \item این کوئری کاری که می‌کند این است که ابتدا
    \lr{ISBN}
    تمامی کتاب‌های
    \lr{springer}
    را پیدا می‌کند. سپس این ست را منهای
    \lr{ISBN}هایی
    می‌کنیم که کاربر قرض گرفته است. این موضوع باعث می‌شود که
    \lr{ISBN}
    کتاب‌هایی را پیدا کنیم که مال
    \lr{springer}
    نیستند و کاربر قرض گرفته است. در نهایت اگر این لیست خالی بود نام کاربر را در جواب
    قرار می‌دهیم. ولی این کوئری غلط است. دلیل آن این است که ممکن است که کاربر تمامی کتاب را قرض گرفته باشد
    ولی در کنار آنها کتاب‌های دیگری از انتشارات دیگری نیز قرض گرفته باشند. این کاربر‌ها از این لیست
    حذف می‌شوند.
    \item این کوئری در داخلی ترین قسمت خود در ابتدا لیست کتاب‌هایی را پیدا می‌کند که انتشارات آنها
    \lr{springer}
    باشد. سپس به ازای هر کتاب، چک میکند که آیا کاربری که در حال چک کردن آن هستیم
    (در بیرونی ترین کوئری)
    کتابی تا به حال این کتاب را قرض گرفته است یا نه. در صورتی که این کتاب را قرض گرفته باشد، از لیست
    حذف می‌شود. در صورتی که حداقل یک کتاب از
    \lr{springer}
    را قرض نگرفته باشند 1 در
    \lr{SELECT}
    بر می‌گردد. پس این کاربر از لیست کاربرها حذف می‌شود. پس در نتیجه این کوئری درست است.
\end{enumerate}
