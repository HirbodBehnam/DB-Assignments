\smalltitle{سوال 2}
\begin{itemize}
    \item در ابتدا کلید‌های کاندید را پیدا می‌کنیم.
    \begin{gather*}
        \{A, B\}\\
        \{E, A\}
    \end{gather*}
    چرا که بستار این صفات برابر
    $R$
    می‌شود و دیگر نمی‌توان این صفات را کاهش داد.
    سپس برای پیدا کردن ابرکلید‌ها به این کلید‌های کاندید می‌توانیم بقیه‌ی صفت‌ها را اضافه کنیم. پس در کل
    $2^2 + 2^2 + 2^2 = 12$
    ابرکلید داریم. این فرمول از اینجا می‌آید که می‌توان در مجموعه‌ی اول و دوم
    $C$ و $D$
    را اضافه کرد. پس در اینجا دو تا
    $2^2$
    حالت داریم. دقت کنید که تا اینجا تمام مجموعه‌ها متفاوت بودند. از طرفی دیگر فرض می‌کنیم که در مجوعه‌ی اول
    $E$
    و در مجموعه‌ی دوم
    $B$
    را اضافه کردیم. حال این دو مجموعه یکی می‌شوند. از اینجا به بعد نیز می‌توان
    $C$ و $D$
    را به این مجموعه اضافه کرد. پس داریم
    $2^2$
    حالت دیگر.
    پس در نتیجه 12 ابرکلید داریم.
    \item در این قسمت نیز در ابتدا کلید‌های کاندید را پیدا می‌کنیم.
    \begin{gather*}
        \{A, E\}\\
        \{D, E\}
    \end{gather*}
    دقیقا با همان استدلال قسمت بالا نیز می‌توان به عدد 12 ابرکلید رسید.
\end{itemize}
