\smalltitle{سوال 1}
\begin{itemize}
    \item ابتدا مجموعه‌ی وابستگی‌های تابعی کاهش‌ناپذیر را پیدا می‌کنیم.
    \begin{gather*}
        F = \{S \rightarrow X, S \rightarrow W, T \rightarrow Y, X \rightarrow Y, XY \rightarrow T, XY \rightarrow U, XY \rightarrow Z \}\\
        \text{Decompose} \implies\\
        F_{opt} = \{S \rightarrow X, S \rightarrow W, T \rightarrow Y, X \rightarrow T, X \rightarrow U, X \rightarrow Z\}
    \end{gather*}
    فرض می‌کنیم که رابطه در
    \lr{1NF}
    است و در یک
    \lr{cell}
    از جدول هیچ مقدار چند متغیره‌ای نداریم. حال کلید کاندید را پیدا می‌کنیم.
    نکته‌ای که در اینجا وجود دارد این است که
    $S$
    هم کلید کاندید هست و هم سوپر کلید. مشخص است که هیچ
    \lr{FD}ای
    نداریم که سمت راست آن
    $S$
    باشد پس رابطه در
    \lr{2NF}
    نیز است.
    اما از طرفی مشخص است که رابطه در
    \lr{3NF}
    نیست. چرا که با اینکه
    $S$
    کلید کاندید است، ولی
    \lr{FD}هایی
    داریم که از
    $T$ و $X$
    شروع می‌شوند. برای درست کردن، این رابطه را به چند رابطه میشکانیم:
    \begin{gather*}
        R_1=(\underline{S}, X, W)\\
        R_2=(\underline{T}, Y)\\
        R_3=(\underline{X}, T, U, Z)
    \end{gather*}
    از آنجا که صفت‌های این رابطه‌ها فقط به کلید‌های رابطه‌هایشان وابستگی تابعی دارند پس این رابطه‌ها
    \lr{BCNF}
    نیز هستند.
    \item ابتدا مجموعه‌ی وابستگی‌های تابعی کاهش‌ناپذیر را پیدا می‌کنیم.
    \begin{gather*}
        F = \{A \rightarrow BC, BC \rightarrow AD, D \rightarrow E \}\\
        \text{Decompose} \implies\\
        F_{opt} = \{A \rightarrow B, A \rightarrow C, BC \rightarrow A, BC \rightarrow D, D \rightarrow E \}
    \end{gather*}
    فرض می‌کنیم که رابطه در
    \lr{1NF}
    است و در یک
    \lr{cell}
    از جدول هیچ مقدار چند متغیره‌ای نداریم. دراینجا نیز سوپر کلید و کلید کاندید
    $A$
    است. چرا که
    $A^+ = R$
    است. از آنجا که تنها یک کلید داریم و
    \lr{composite}
    نیست پس در
    \lr{2NF}
    هستیم. اما مثل قسمت قبل می‌توان نتیجه گرفت که در
    \lr{3NF}
    نیستیم. برای اینکه به
    \lr{3NF}
    برویم به چند رابطه، رابطه را میشکانیم.
    \begin{gather*}
        R_1=(\underline{A}, B, C)\\
        R_2=(\underline{B, C}, D)\\
        R_3=(\underline{D}, E)
    \end{gather*}
    حال باید
    $R_2$
    را چک کنیم که آیا ستون‌هایش وابستگی به
    $\underline{BC}$
    دارند یا خیر. که همان طور که از
    $F_{opt}$
    مشخص است
    $D$
    فقط به
    $BC$
    وابسته است پس در
    \lr{BCNF}
    هستیم.
\end{itemize}
